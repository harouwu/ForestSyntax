\documentclass[10pt,twoside,a4paper]{article}

\usepackage[margin=.7in]{geometry}

\usepackage[utf8x]{inputenc}
\usepackage{amsmath, amsthm, amssymb, amsbsy}
\usepackage{microtype}
\usepackage{color}
\usepackage{caption}
\usepackage{graphicx}
\usepackage{hyperref}
\usepackage{thmtools}
\usepackage{wrapfig}
\usepackage{stmaryrd}
\usepackage{listings}
\usepackage{cancel}
\usepackage[all]{xy}
\usepackage{mathtools}
\usepackage{array}

\theoremstyle{theorem}
\newtheorem{theorem}{Theorem}[section]
\theoremstyle{lemma}
\newtheorem{lemma}{Lemma}[section]
\theoremstyle{property}
\newtheorem{property}{Property}[section]
\theoremstyle{definition}
\newtheorem{definition}{Definition}[section]
\theoremstyle{assumption}
\newtheorem{assumption}{Assumption}[section]

\def\fst{\pi_1}
\def\snd{\pi_2}
\def\id{\emptyset}

\title{Incremental Forest Draft}
\author{Yiming Wu}
\begin{document}

\maketitle

\section{Model of Incremental Forest}

Incremental Forest aims to improve efficiency of Forest. It enables Forest to respond to changes on file systems and environment. 
The Incremental Forest detects changes and apply them only to the corresponding metadata and representation instead of reload the whole file system again. \\

In static Forest, we only have $\mathtt{load}$ function that loads certain file system $F$ under path $r$ into memory as representation $v$ and metadata $d$. The original Forest model can be abstracted like:

\begin{displaymath}
	F,\varepsilon \xrightarrow{\;\;\;load\;\;\;} (v,d) 
\end{displaymath}

This model fails to meet the requirement of real world application for that it has to reload the whole file system again if either file system or environment changes. This makes implementation cumbersome.\\

To improve efficiency of Forest, we introduce a delta-load function $\mathtt{load}_\Delta$ that takes in changes(delta) on file system and enviroment, $\delta_F$ and $\delta_\varepsilon$ respectively. Then $\mathtt{load}_\Delta$ generates a squence of changes $\Delta_v = \delta_{v1} \cdot \delta_{v2} \dots \delta_{vn}$ and $\Delta_d = \delta_{d1} \cdot \delta_{d2} \dots \delta_{dn}$ that will be use to update representation and metadata in memory.
\begin{displaymath}
	\xymatrix@=2cm{
	F,\, \varepsilon \ar@{->}[r]^{\mathtt{load}}
	\ar@{..>}[d]_{\delta_F, \delta_\varepsilon}="a"
	& (v,d)\ar@{..>}[d]^{\Delta_v, \Delta_d}="b"\\
	%
	F',\, \varepsilon' \ar@{->}[r]^{\mathtt{load}}
	& (v',d')
	\ar@{=>}^{\mathtt{load_\Delta}}"a";"b"
	}
\end{displaymath}

\begin{displaymath}
\Delta_v = \delta_{v1} \cdot \delta_{v2} \dots \delta_{vn}	\quad 	\Delta_d = \delta_{d1} \cdot \delta_{d2} \dots \delta_{dn}
\end{displaymath}

The square figure above shows a abstracted model of Incremental Forest. Some parameters like specification $s$ are hidden for simplicity. The $(v',d')$ is the result of loading the new file system again. In Incremental Forest, we guarantee that we will get same in-memory representation if we apply the sequence of changes $\Delta_v$ and $\Delta_d$ to original $(v,d)$.\\

In the following section,we will first give out syntax and semantic of Incremental Forest. Then we will propose and prove properties and theorum of correctness of Incremental Forest.

\newpage

\section{Syntax definition}

In this section, we will first discuss definition of Incremental Forest syntax. Then we will introduce incremental semantic under different specifications.\\

$\boxed{Static\; Forest\; definitions}$
\begin{align*}
	& Specification ~~s	& ::= 	& ~k^{\pi_1}_{\pi_2} ~|~ e::s ~|~ \langle x : s_1, s_2 \rangle ~|~ \{s\mid x\in e\} ~|~ s = P(e) ~|~ s? \\
	& Type ~~t & ::= 	& ~\mathtt{String} ~|~ t_1 \times t_2 ~|~ t~\mathtt{set} ~|~ \mathtt{Maybe}~t ~|~ \mathtt{()} ~|~ \mathtt{Bool}\\
	& Environment ~~\varepsilon & ::= & ~\emptyset ~|~ \varepsilon, x\mapsto v
\end{align*}

$\boxed{incremental\;Forest\;definitions}$
%!!!!!laignment problem
\begin{align*}
	& Change ~~\delta & ::= 	& ~\leadsto v ~|~ \delta_1 \cdot \delta_2 ~|~ \fst\,\delta_1 ~|~ \snd\,\delta_2 ~|~ \delta_1? ~|~ \delta_1 \otimes \delta_2 ~|~ f \leftarrow \delta_1\\
	& 		& |		&~ \mathtt{add}(n, v) ~|~ \mathtt{del}(n, v) ~|~ \mathtt{mod}(n, \delta_1) ~|~ \emptyset\\
	& Update~on~files ~~\delta_F 	& ::= 	& ~\mathtt{addFile}(r,\omega) ~|~ \mathtt{rmvFile}(r) ~|~ \mathtt{chgAttri}(r,a') ~|~ \delta_1 \cdot \delta_2 ~|~ \emptyset\\
	& Update~on~environments ~~\delta_\varepsilon & ::= & ~\emptyset ~|~ \delta_\varepsilon, x \mapsto (v,\delta_v) \\
	%& Update ~~u 	  & ::=		& ~ \emptyset ~|~ u \cdot \delta_F ~|~ u \cdot \delta_\varepsilon
\end{align*}

Static Forest definition can be found in previous Forest paper. In incremental syntax, we define changes on in-memory data $\delta$, updates on file system $\delta_F$ and updates on environment $\delta_\varepsilon$. \\

In syntax of changes $\delta$, $\leadsto v$ means overwriting the value with $v$; 
$\delta_1 \cdot \delta_2$ means combination of changes; 
$\fst\,\delta_1$ means apply the change $\delta_1$ to the first element of the pair, $\snd\,\delta_1$ means apply the change $\delta_1$ to the second element of the pair; 
$\delta_1?$ means apply the change and find out either this file system that is undefined at the current path, or a file system containing the current path and satisfying the specification; 
the $\delta_1 \otimes \delta_2$ equals to $\fst\,\delta_1 \cdot \snd\,\delta_2$, which can be applied to change two elements of a pair separately; 
The other three $\mathtt{add}(n, v) ~|~ \mathtt{del}(n, v) ~|~ \mathtt{mod}(n, \delta_1)$ is for applying changes for elements in the map from the comprehension specification; 
$\emptyset$ is for empty sequence, which means doing nothing. \\

Updates on file system is abstracted to adding a file with content $\omega$ to path $r$ as $\mathtt{addFile}(r,\omega)$; deleting a file from path $r$ as $\mathtt{rmvFile}(r)$; changing attributes of a file as $\mathtt{chgAttri}(r,a')$; combinations of updates as $\delta_1 \cdot \delta_2$ and finally, doing nothing as $\emptyset$.\\

For Forest environment, $\varepsilon: V\!ars \mapsto V\!als$, it maps variables to values. Environment update $\delta_\varepsilon: V\!ars \mapsto V\!als \times \delta_v$ maps from variables to product of values and its change. Thus for any variable $x$, we have:
\begin{eqnarray*}
	\varepsilon (x) &=& v\\
	\delta_\varepsilon (x) &=& (\varepsilon(x), \delta_x)
\end{eqnarray*}

For example, in Adaptive Forest, environment always contain a variable r which maps to the current path. We define environment before update $\varepsilon_{old}$, alias \emph{old} $\varepsilon$; environment after update $\varepsilon_{new}$, alias \emph{new} $\varepsilon$. Given any updates $\delta_\varepsilon$:
\begin{align*}
	\varepsilon_{old}(x) =& ~v\\
						  & ~where ~\delta_\varepsilon (x) = (v, \delta_v)\\
	\varepsilon_{new}(x) =& ~v'\\
						  & ~where ~\delta_\varepsilon (x) = (v, \delta_v)\\
						  & ~and ~v' = \delta_v v
\end{align*}

We also define equivalance of two environments $\varepsilon_1$ and $\varepsilon_2$
\begin{displaymath}
	\varepsilon_1 = \varepsilon_2
\end{displaymath}
\begin{displaymath}
	where \quad \mathtt{dom}(\varepsilon_1) = \mathtt{dom}(\varepsilon_2) ~\wedge~ \forall x \in \mathtt{dom}(\varepsilon_1), \quad \delta_\varepsilon(x) = (x, \emptyset)
\end{displaymath}

After we define all the syntax of Incremental Forest, we can move on to the semantic part.

\section{Semantic of Incremental Forest}

We sill discuss semantic of Incremental Forest under different specification in this section.\\

$\boxed{s ::= k^{\pi_1}_{\pi_2}}$

\begin{displaymath}
	\frac{F(r) = (a, \mathtt{File}(\omega))}
	{\varepsilon, k \vdash \mathtt{load}~F \rhd (\omega, (True, a))}
\end{displaymath}

%$\boxed{\delta_\varepsilon, k \vdash \mathtt{load}_\Delta (F,v,d):}$
\begin{align*}
	&\delta_\varepsilon, k \vdash \mathtt{load}_\Delta (F,v,d):
\end{align*}
\begin{align*}
	\delta_F					 &\rhd& ~(\leadsto \omega', \leadsto(True, a))			& &~if ~\varepsilon_{new}(r) \not= \varepsilon_{old}(r) \\
	\mathtt{addFile}(r',\omega') &\rhd& ~(\leadsto \omega', \leadsto(True, a))			& &~if ~\varepsilon_{new}(r) = r' \\
								 &\rhd& ~(\emptyset,\emptyset)								& &~otherwise \\
	\mathtt{rmvFile}(r') 		 &\rhd& ~(\leadsto `` '', \leadsto(False, default))			& &~if ~\varepsilon_{new}(r) = r' \\
								 &\rhd& ~(\emptyset,\emptyset)								& &~otherwise \\
	\mathtt{addAttri}(r', a')	 &\rhd& ~(\emptyset, \snd(\leadsto a'))				& &~if ~\varepsilon_{newtheorem}(r) = r' \\
								 &\rhd& ~(\emptyset,\emptyset)								& &~otherwise	\\
	\emptyset					 &\rhd& ~(\emptyset,\emptyset)								& &~ if ~\varepsilon_{new} = \varepsilon_{old}\\
								 &\rhd& ~(\leadsto \omega', \leadsto (Ture, a))				& &~ if ~\varepsilon_{new} \not= \varepsilon_{old} ~\wedge ~\varepsilon_{new}, k \vdash \mathtt{load}~F \rhd (\omega', (True, a))
\end{align*}

In incremental Forest, $\delta_\varepsilon$ contains the current path value in the variable $r$. 
If the path of file that Incremental Forest should load has changed, then Incremental Forest will reload it with highest priority.
When an update occurs in the file system, incremental Forest will change the representation and metadata if the update happens at the current path. 
In the rest of the section, we will use the $\varepsilon_{old}(r)$ or simply $r$ to get the old path.\\

$\boxed{Combination ~ of ~ Updates}$

\begin{displaymath}
	\frac{\begin{array}{c}
		\delta_\varepsilon, k \vdash \mathtt{load}_\Delta (F,v,d)~ \delta_{F_1} \rhd (\delta_{v_1}, \delta_{d_1})\\
		\delta_\varepsilon, k \vdash \mathtt{load}_\Delta (\delta_{F_1} F,~\delta_{v_1} v,~\delta_{d_1} d)~ \delta_{F_2} \rhd (\delta_{v_2}, \delta_{d_2})
	\end{array}}
	{\delta_\varepsilon, k \vdash \mathtt{load}_\Delta (F,v,d)~ (\delta_{F_1} \cdot \delta_{F_2}) \rhd ((\delta_{v_2} \cdot \delta_{v_1}), (\delta_{d_2} \cdot \delta_{d_1}))}
\end{displaymath}

\begin{displaymath}
	\frac{\begin{array}{c}
		{\delta_\varepsilon}' = (\delta_\varepsilon, x \mapsto (m, \delta_m))\\
		(\delta_\varepsilon, x \mapsto (m, \delta_m)), k \vdash \mathtt{load}_\Delta (F,v,d)~ \delta_F \rhd (\delta_v, \delta_d)
	\end{array}}
	{{\delta_\varepsilon}', k \vdash \mathtt{load}_\Delta (F,v,d)~ \delta_F \rhd (\delta_v, \delta_d)}
\end{displaymath}

In Incremental Forest, if an update can be splited into separate updates, the result stays the same between loading separate updates step by step or load them as one update.\\

$\boxed{e::s}$

\begin{displaymath}
	\frac{
		(\varepsilon, r \mapsto \llbracket r/e \rrbracket^\varepsilon_{Path}) , s \vdash \mathtt{load}~F \rhd (v, d)
	}
	{
		\varepsilon, e::s \vdash \mathtt{load}~F \rhd (v,d)
	}
\end{displaymath}

$\boxed{Incremental~~e::s}$
%\begin{displaymath}
%	\frac{\begin{array}{c}
%		\llbracket r/e \rrbracket^{\varepsilon_{old}}_{Path} \not= \llbracket r/e \rrbracket^{\varepsilon_{new}}_{Path}\\
%		(\delta_\varepsilon, r \mapsto (r, \leadsto \llbracket r/e \rrbracket^{\varepsilon_{new}}_{Path})) , s \vdash \mathtt{load}_\Delta (F,v,d)~ \delta_F \rhd (\delta_v,\delta_d)
%	\end{array}}
%	{
%		\delta_\varepsilon, e::s \vdash \mathtt{load}_\Delta (F,v,d)~ \delta_F \rhd (\delta_v,\delta_d)
%	}
%\end{displaymath}

\begin{displaymath}
	\frac{\begin{array}{c}
		\llbracket r/e \rrbracket^{\varepsilon_{old}}_{Path} = \llbracket r/e \rrbracket^{\varepsilon_{new}}_{Path}\\
		(\delta_\varepsilon \setminus r, r \mapsto (\llbracket r/e \rrbracket^{\varepsilon_{new}}_{Path},\emptyset)) , s \vdash \mathtt{load}_\Delta (F,v,d)~ \delta_F \rhd (\delta_v,\delta_d)
	\end{array}}
	{
		\delta_\varepsilon, e::s \vdash \mathtt{load}_\Delta (F,v,d)~ \delta_F \rhd (\delta_v,\delta_d)
	}
\end{displaymath}
\begin{displaymath}
	\frac{\begin{array}{c}
		\llbracket r/e \rrbracket^{\varepsilon_{old}}_{Path} \not= \llbracket r/e \rrbracket^{\varepsilon_{new}}_{Path} ~\land \\
		\varepsilon_\delta, \llbracket r/e \rrbracket^{\varepsilon_{new}}_{Path} , s \vdash \mathtt{load}(F')~ u \rhd (v,d) ~~ where ~ F' = \delta_F F
	\end{array}}
	{
		\delta_\varepsilon, r, e::s \vdash \mathtt{load}_\Delta (F,v,d)~ u \rhd (\leadsto v,\leadsto d)
	}
\end{displaymath}

In Incremental Forest, path specification $e\!::\!s$ will create a new path but may not result in the same path in old environment and new environment. 
We solve this by always applying the new path in the new environment to the path variable $r$. 
When the path changes, Incremental Forest will record the change of path. This change in file path will finally trigger reload semantic in constant specification.
When the path keeps the same, one tricky part here is we don't change the path variable. Instead, we add a new path variable into the $\delta_environment$ so that Incremental Forest will not produce any extra environment change itself.\\
%The delta environment $(\delta_\varepsilon \setminus r, r \mapsto (\llbracket r/e \rrbracket^{\varepsilon_{new}}_{Path}),id)$ means Incremental Forest will abandon the old file path and focus on the new file path.\\

$\boxed{s = \langle x : s_1, s_2 \rangle}$

\begin{displaymath}
	\frac{\begin{array}{c}
		\varepsilon, s_1 \vdash \mathtt{load}~F \rhd (v_1,d_1)\\
		(\varepsilon, x \mapsto v_1, x_{md} \mapsto v_2), s_2 \vdash \mathtt{load}~F \rhd (v_2,d_2)\\
		b = \mathtt{valid}(d_1) \land \mathtt{valid}(d_2)
	\end{array}}
	{\varepsilon, \langle x : s_1, s_2 \rangle \vdash \mathtt{load}~F \rhd ((v_1, v_2),(b,(d_1,d_2))) }
\end{displaymath}

$\boxed{Incremental~~s = \langle x : s_1, s_2 \rangle}$

\begin{displaymath}
	\frac{\begin{array}{c}
		\delta_\varepsilon, s_1 \vdash \mathtt{load}_\Delta (F,v_1,d_1)~ \delta_F \rhd (\delta_{v_1},\delta_{d_1})\\
		(\delta_\varepsilon, x \mapsto (\delta_{v_1} v_1, \emptyset), x_{md} \mapsto (\delta_{d_1} d_1, \emptyset)), s_2 \vdash \mathtt{load}_\Delta (F,v_2,d_2)~ \delta_F \rhd (\delta_{v_2},\delta_{d_2})\\
		valid_{12}({d_1}',{d_2}') = valid({d_1}') \wedge valid({d_2}')
	\end{array}}
	{\delta_\varepsilon, \langle x:s_1,s_2 \rangle \vdash \mathtt{load}~ (F,(v_1,v_2),(b,(d_1,d_2)))~ u \rhd (\delta_{v_1} \otimes \delta_{v_2},valid_{12} \leftarrow (\delta_{d_1} \otimes \delta_{d_2})) }
\end{displaymath}

If reader find it difficult to understand $(\delta_{v_1} \otimes \delta_{v_2},\leadsto valid_{12} \otimes (\delta_{d_1} \otimes \delta_{d_2}))$, it can be viewed as apply corresponding change to $((v_1,v_2),(b,(d_1,d_2)))$ and recheck the correctness of new metadata $({d_1}', {d_2}')$\\

$\boxed{s = \mathtt{P}(e)}$

\begin{displaymath}
	\frac
	{b = \llbracket e \rrbracket^{\varepsilon}_{Bool}}
	{\varepsilon, \mathtt{P}(e) \vdash \mathtt{load}~F \rhd ((),(b,()))}
\end{displaymath}

$\boxed{Incremental~~s = P(e)}$

\begin{displaymath}
	\frac{ \varepsilon_{old} = \varepsilon_{new}}
	{\delta_\varepsilon, P(e) \vdash \mathtt{load}~ (F,v,d) ~\delta_F \rhd (\emptyset, \emptyset)}
\end{displaymath}

\begin{displaymath}
	\frac{ \varepsilon_{old} \not= \varepsilon ~\wedge~  b = \llbracket e \rrbracket^{\varepsilon_{new}}_{Bool}}
	{\delta_\varepsilon, P(e) \vdash \mathtt{load}~ (F,v,d) ~\delta_F \rhd (\emptyset,\fst(\leadsto b))}
\end{displaymath}

If nothing has changed in the environment, Incremental Forest will spare the labor of testing the bool expresiion. Otherwise, Incremental Forest will test the expression under new environment.\\

$\boxed{Incremental~~s = s?}$

\begin{displaymath}
	\frac{
		r' = \varepsilon_{new}(r) \quad r \notin \mathtt{dom}(F) \quad r' \notin \mathtt{dom}(\delta_F~F)
	}
	{\delta_\varepsilon, s? \vdash \mathtt{load}_\Delta (F,v,d)~ u \rhd (\emptyset, \emptyset)}
\end{displaymath}

\begin{displaymath}
	\frac{
		r' = \varepsilon_{new}(r)  \quad r \in \mathtt{dom}(F) \quad r' \notin \mathtt{dom}(\delta_F~F)
	}
	{\delta_\varepsilon, s? \vdash \mathtt{load}_\Delta (F,v,d)~ u \rhd (\leadsto Nothing,\leadsto (True,Nothing))}
\end{displaymath}

\begin{displaymath}
	\frac{\begin{array}{c}
	r' = \varepsilon_{new}(r) \quad r \notin \mathtt{dom}(F) \quad r' \in \mathtt{dom}(\delta_F~F), \quad
		\varepsilon_{new}, s \vdash \mathtt{load}~ (\delta_F~F) \rhd (v',d')
	\end{array}}
	{\delta_\varepsilon, s? \vdash \mathtt{load}_\Delta (F,v,d)~ u \rhd (\leadsto Just~v',\leadsto (valid(d'),Just~d'))}
\end{displaymath}

\begin{displaymath}
	\frac{\begin{array}{c}
	r' = \varepsilon_{new}(r) \quad r \in \mathtt{dom}(F) \quad r' \in \mathtt{dom}(\delta_F~F), \quad
		\delta_\varepsilon, s \vdash \mathtt{load}_\Delta (F,v,d)~ \delta_F \rhd (\delta_v,\delta_d)
	\end{array}}
	{\delta_\varepsilon, s? \vdash \mathtt{load}_\Delta (F,v,d)~ \delta_F \rhd (\delta_v,\delta_d)}
\end{displaymath}

Incremental Forest will check whether the new path $r'$ is defined under the new file system $\delta_F~F$ and perform the same operation as static Forest does.\\

$\boxed{s = [ s \mid x \in e]}$

\begin{displaymath}
	\frac{\begin{array}{c}
		\llbracket e \rrbracket^{\varepsilon}_{\{Path\}} = \{n_1,n_2,\dots,n_k\}\\
		\forall i \in \{1,2,\dots,k\}, (\varepsilon, r \mapsto r/n_i,s) \vdash \mathtt{load}~F \rhd (v_i, d_i)\\
		vs ~\mathtt{maps} ~n_i \to v_i, ~ds ~\mathtt{maps} ~n_i \to d_i, ~b = \bigwedge_{i=1}^k valid(d_i) 
	\end{array}}
	{\varepsilon, [ s \mid x \in e] \vdash \mathtt{load}~F \rhd (vs,(b,ds)) }
\end{displaymath}

To make Forest incremental, we make some changes in static comprehension specification. First, we abandon arbitrary type of comprehension and restrict it to be set of filenames $n$. Second, we write the loading path for every filename explicitly as $r/n_i$ while static Forest exploit $e\!::\!s$ implicitly to help it accomplish that. Third, we change the data structure of $vs$ and $ds$ as $\mathtt{map}$ so that we can easily find representation $v_i$ and metadata $d_i$ of certain filename $n_i$. 
\begin{eqnarray*}
	vs(n) = v \\
	ds(n) = v 
\end{eqnarray*}

All these changes are made to preserve the atomic property of Incremental Forest so that we can get all the information we need in one operation.\\

$\boxed{Incremental~~ s = [ s \mid x \in e]}$
\begin{displaymath}
	\frac{\begin{array}{c}
		\Delta_v = \emptyset, ~\Delta_d = \emptyset\\
		\forall n_i \in \llbracket e \rrbracket^{\varepsilon_{old}}_{\{Path\}} \cup \llbracket e \rrbracket^{\varepsilon_{new}}_{\{Path\}}~~
		\Delta_v = \Delta_v \cdot \mathtt{changeVOf}(n_i),~ \Delta_d = \Delta_d \cdot \mathtt{changeDOf}(n_i)\\
		valid_{all} (ds') = \bigwedge_{i=1}^l valid(d_i'), ~d_i' \in ds'
	\end{array}}
	{\delta_\varepsilon, r, [ s \mid x \in e] \vdash \mathtt{load}_\Delta~ (F,vs,ds)~ \delta_F \rhd (\Delta_v, valid_{all} \leftarrow \Delta_d) }
\end{displaymath}

Since changes on the filename set is arbitrary, here we design a $\Delta_v$ to record all the changes on representation and $\Delta_d$ to record all the changes on metadata. Incremental Forest will iterate all of the filenames, find out how should we change the representation and metadata of each file and combine them together. The rule to determine the change of a single file is dicribed by $\mathtt{changeVOf}$ and $\mathtt{changeDOf}$\\ 

$\boxed{\mathtt{changeVOf} ~ and ~ \mathtt{changeDOf}}$

\begin{displaymath}
	\frac
	{\forall n \not\in \llbracket e \rrbracket^{\varepsilon_{new}}_{\{Path\}},~ v = vs(n), ~d = ds(n)}
	{\mathtt{changeVOf}(n) = \mathtt{del}((n,v)), ~~\mathtt{changeDOf}(n) = \mathtt{del}((n,d))}
\end{displaymath}
If the filename $n$ doesn't exist in new environment, Incremental Forest delete corresponding representation and metadata from the set.

\begin{displaymath}
	\frac{\begin{array}{c}
	\forall n \in \llbracket e \rrbracket^{\varepsilon_{new}}_{\{Path\}} \setminus \llbracket e \rrbracket^{\varepsilon_{old}}_{\{Path\}}, ~ \varepsilon_{new}, r/n,s \vdash \mathtt{load}~F \rhd (v, d)
	\end{array}}
	{\mathtt{changeVOf}(n) = \mathtt{add}((n,v)), ~~\mathtt{changeDOf}(n) = \mathtt{add}((n,d))}
\end{displaymath}
If the filename $n$ doesn't exist in old environment, Incremental Forest add corresponding representation and metadata to the set.

\begin{displaymath}
	\frac{\begin{array}{c}
	\forall n \in \llbracket e \rrbracket^{\varepsilon_{new}}_{\{Path\}} \cap \llbracket e \rrbracket^{\varepsilon_{old}}_{\{Path\}}, ~ v = vs(n), ~d = ds(n)\\
	(\delta_\varepsilon, r \mapsto (r/n, \emptyset)), s \vdash \mathtt{load}_\Delta (F,v,d)~ u \rhd (\delta_v,\delta_d)
	\end{array}}
	{\mathtt{changeVOf}(n) = \mathtt{mod}((n,v), \snd\delta_v), ~~\mathtt{changeDOf}(n) = \mathtt{mod}((n,d), \snd\delta_d)}
\end{displaymath}
If the filename $n$ is preserved in both environments, Incremental Forest delta-load the file recursively. One thing to notice here is that we append the name $n$ directly to the filepath $r$. Because in either old or new environment, the expression $e$ can always be evaluated to this $r/n$ filepath, in this level of recursion, we don't need to record the change in filepath. This makes the delta environment clear of unexpected change on variables.

\newpage
\section{Proof}
In this section, we will discuss about assumptions and theorems of Incremental Forest.
\begin{assumption}[Update Sequence]
	Incremental Forest takes in a sequence of updates consists of 
	\begin{displaymath}
		\mathtt{addFile}(r,\omega) ~|~ \mathtt{rmvFile}(r) ~|~ \mathtt{chgAttri}(r,a')
	\end{displaymath}
	Or empty sequence $\emptyset$.
\end{assumption}

Complex updates can be constructed from atomic updates.

\begin{assumption}[Filesystem is a tree]
	Incremental Forest views symbolic link as a normal file. \\
	not necessarily needed.
\end{assumption}

\begin{theorem}[Soundness]
	Incremental Forest satisfied the rule that
	\begin{align*}
		& \varepsilon, s \vdash \mathtt{load}~ F~ \rhd (v,d)\\
		& \delta_\varepsilon, s \vdash \mathtt{load}~ (F,v,d)~ \delta_F \rhd (\delta_v,\delta_d)\\
		& \varepsilon_{new}, s \vdash \mathtt{load}~ (\delta_F~F)~ \rhd (v',d')
	\end{align*}
	\begin{displaymath}
		v' = \delta_v v, \quad d' = \delta_d d
	\end{displaymath}
\end{theorem}

\begin{proof}
	By induction\\
	\textbf{Case: } $s = k$\\
	\begin{enumerate}
	\item
	By definition, if $\varepsilon_{old}(r) \not= \varepsilon_{new}(r)$,
	\begin{align*}
			& \varepsilon, s \vdash \mathtt{load}~ F~ \rhd (v,d)\\
			& \delta_\varepsilon, s \vdash \mathtt{load}~ (F,v,d)~ \delta_F \rhd (\leadsto \omega', \leadsto (True, a))\\
			& \varepsilon_{new}, s \vdash \mathtt{load}~ (\delta_F~F)~ \rhd (\omega', (True, a))
	\end{align*}
	\begin{displaymath}
		\leadsto \omega' ~v = \omega' \quad \leadsto (True, a) ~d = (True, a)
	\end{displaymath}
	The Soundness Theorem holds for the $\varepsilon_{old}(r) \not= \varepsilon_{new}(r)$ case here. \\

	\item
	By definition, if $\delta_F = \mathtt{addFile}(r',\omega')$, here we use $r$ to represent $\varepsilon_{old}(r)$ \\
	\textbf{Case: } $r = r'$
	\begin{align*}
			& \varepsilon, s \vdash \mathtt{load}~ F~ \rhd (v,d)\\
			& \delta_\varepsilon, s \vdash \mathtt{load}~ (F,v,d)~ \delta_F \rhd (\leadsto \omega', \leadsto (True, a))\\
			& \varepsilon_{new}, s \vdash \mathtt{load}~ (\delta_F~F)~ \rhd (\omega', (True, a))
	\end{align*}
	\begin{displaymath}
		\leadsto \omega' ~v = \omega' \quad \leadsto (True, a) ~d = (True, a)
	\end{displaymath}
	The Soundness Theorem holds for the $r = r'$ case here. \\
	\textbf{Case: } $r \not= r'$
	\begin{align*}
			& \varepsilon, s \vdash \mathtt{load}~ F~ \rhd (v,d)\\
			& \delta_\varepsilon, s \vdash \mathtt{load}~ (F,v,d)~ \delta_F \rhd (\emptyset, \emptyset)\\
			& \varepsilon_{new}, s \vdash \mathtt{load}~ (\delta_F~F)~ \rhd (v,d)
	\end{align*}
	\begin{displaymath}
		\emptyset ~v = v \quad \emptyset ~d = d
	\end{displaymath}
	The Soundness Theorem holds for the $r \not= r'$ case here.\\
	With same method, we can easily prove that Soundness holds for $\mathtt{rmvFile}(r')$, $\mathtt{addAttri}(r', a')$ and $\emptyset$ scenarios. Thus we can conclude that Soundness holds for the constant specification case.
	\end{enumerate}
	
	\textbf{Case: } $e::s$\\
	\begin{enumerate}
	\item 
	When the generated path $\llbracket r/e \rrbracket^{\varepsilon_{new}}_{Path}$ keeps the same in both old and new environment, by induction, we assume that Soundness holds for $(\delta_\varepsilon \setminus r, r \mapsto (\llbracket r/e \rrbracket^{\varepsilon_{new}}_{Path}, \emptyset)) , s \vdash \mathtt{load}_\Delta (F,v,d)~ \delta_F \rhd (\delta_v,\delta_d)$, which means.
	\begin{align*}
			& (\varepsilon, r \mapsto \llbracket r/e \rrbracket^{\varepsilon_{new}}_{Path}), s \vdash \mathtt{load}~ F~ \rhd (v,d)\\
			& (\delta_\varepsilon \setminus r, r \mapsto (\llbracket r/e \rrbracket^{\varepsilon_{new}}_{Path}, \emptyset)) , s \vdash \mathtt{load}_\Delta (F,v,d)~ \delta_F \rhd (\delta_v,\delta_d)\\
			& (\varepsilon, r \mapsto \llbracket r/e \rrbracket^{\varepsilon_{new}}_{Path}), s \vdash \mathtt{load}~ F~ \rhd (\delta_v~v,\delta_d~d)
	\end{align*}
	By definition of $e::s$ static load and incremental load, we will have
	\begin{align*}
			& \varepsilon, e::s \vdash \mathtt{load}~ F~ \rhd (v,d)\\
			& \delta_\varepsilon, e::s \vdash \mathtt{load}_\Delta (F,v,d)~ \delta_F \rhd (\delta_v,\delta_d)\\
			& \varepsilon, e::s \vdash \mathtt{load}~ F~ \rhd (\delta_v~v,\delta_d~d)
	\end{align*}

	\item 
	When the generated path changes, by induction, we assume that Soundness holds for $(\delta_\varepsilon \setminus r, r \mapsto (r, \llbracket r/e \rrbracket^{\varepsilon_{new}}_{Path})) , s \vdash \mathtt{load}_\Delta (F,v,d)~ \delta_F \rhd (\delta_v,\delta_d)$, which means.
	\begin{align*}
			& (\varepsilon, r \mapsto \llbracket r/e \rrbracket^{\varepsilon_{new}}_{Path}), s \vdash \mathtt{load}~ F~ \rhd (v,d)\\
			& (\delta_\varepsilon \setminus r, r \mapsto (\llbracket r/e \rrbracket^{\varepsilon_{new}}_{Path}, \emptyset)) , s \vdash \mathtt{load}_\Delta (F,v,d)~ \delta_F \rhd (\delta_v,\delta_d)\\
			& (\varepsilon, r \mapsto \llbracket r/e \rrbracket^{\varepsilon_{new}}_{Path}), s \vdash \mathtt{load}~ F~ \rhd (\delta_v~v,\delta_d~d)
	\end{align*}
	By definition of $e::s$ static load and incremental load, we will have
	\begin{align*}
			& \varepsilon, e::s \vdash \mathtt{load}~ F~ \rhd (v,d)\\
			& \delta_\varepsilon, e::s \vdash \mathtt{load}_\Delta (F,v,d)~ \delta_F \rhd (\delta_v,\delta_d)\\
			& \varepsilon, e::s \vdash \mathtt{load}~ F~ \rhd (\delta_v~v,\delta_d~d)
	\end{align*}
	Thus we can conclude that Soundness holds for the focus specification case.
	\end{enumerate}
	
	\textbf{Case: } $s = \langle x : s_1, s_2 \rangle$\\
	\begin{enumerate}	
	\item
	By definition
	\begin{align*}
			& \varepsilon, \langle x : s_1, s_2 \rangle \vdash \mathtt{load}~ F~ \rhd ((v_1, v_2), (b, (d_1, d_2)))\\
			& \delta_\varepsilon, \langle x : s_1, s_2 \vdash \mathtt{load}~ (F,v,d)~ \delta_F \rhd (\delta_v, \delta_d)\\
			& \varepsilon_{new}, \langle x : s_1, s_2 \rangle \vdash \mathtt{load}~ (\delta_F~F)~ \rhd ((v_1', v_2'), (b', (d_1', d_2')))
	\end{align*}
	By induction, Soundness holds for $s_1$.
	\begin{displaymath}
		\delta_\varepsilon, s_1 \vdash \mathtt{load}_\Delta (F,v_1,d_1)~ u \rhd (\delta_{v_1},\delta_{d_1})
	\end{displaymath}
	\begin{displaymath}
		\delta_{v_1} ~v_1 = v_1' \quad \delta_{d_1} ~d_1 = d_1'
	\end{displaymath}
	Because the delta environment $(\delta_\varepsilon, x \mapsto (v_1,\delta_{v_1}), x_{md} \mapsto (d_1,\delta_{d_1}))$ will keep the record of update of $v_1$ and $d_1$, Soundness also holds for $s_2$ so that
	\begin{displaymath}
		(\delta_\varepsilon, x \mapsto (v_1,\delta_{v_1}), x_{md} \mapsto (d_1,\delta_{d_1})), s_2 \vdash \mathtt{load}_\Delta (F,v_2,d_2)~ u \rhd (\delta_{v_2},\delta_{d_2})
	\end{displaymath}
	\begin{displaymath}
		\delta_{v_2} ~v_2 = v_2' \quad \delta_{d_2} ~d_2 = d_2'
	\end{displaymath}
	To apply changes on data, we use $\otimes$ to apply two changes. The result should be the same with $(v_1', v_2')$ and $(d_1', d_2')$.
	\begin{displaymath}
		\delta_{v_1} \otimes \delta_{v_2} ~(v_1, v_2) = (v_1', v_2') \quad \delta_{v_1} \otimes \delta_{d_2} ~(d_1, d_2) = (d_1', d_2')
	\end{displaymath}
	Finally $valid_{12} \leftarrow$ will validate the two metadata and change the original one. The result of it goes the same with the $b'$. Thus we can conclude that Soundness holds for the dependant pair specification case.
	\end{enumerate}

	\textbf{Case: } $s = P(e)$\\
	\begin{enumerate}
	\item
	By definition
	\begin{align*}
		& \varepsilon, P(e) \vdash \mathtt{load}~ F~ \rhd ((),(b,()))\\
		& \delta_\varepsilon, P(e) \vdash \mathtt{load}~ (F,v,d)~ \delta_F \rhd (\delta_v, \fst(\leadsto b'))\\
		& \varepsilon_{new}, P(e) \vdash \mathtt{load}~ (\delta_F~F)~ \rhd ((),(b'',()))
	\end{align*}
	In the equations, 
	\begin{displaymath}
		b' = \llbracket e \rrbracket^{\varepsilon_{new}}_{Bool} \quad b'' = \llbracket e \rrbracket^{\varepsilon_{new}}_{Bool}
	\end{displaymath}
	The $b'$ and $b''$ holds the same value. Thus we can conclude that Soundness holds for the predicate specification case.
	\end{enumerate}

	\textbf{Case: } $s = s_1?$\\
	\begin{enumerate}
	\item
	When $r \notin \mathtt{dom}(F) \quad r' \notin \mathtt{dom}(\delta_F~F)$
	\begin{align*}
		& \varepsilon, s_1? \vdash \mathtt{load}~ F~ \rhd (Nothing,(True, Nothing))\\
		& \varepsilon_{new}, s_1? \vdash \mathtt{load}~ (\delta_F~F)~ \rhd (Nothing,(True, Nothing))
	\end{align*}
	The delta load function will generate $(\emptyset, \emptyset)$, which satisfied the Soundness.

	\item
	When $r \in \mathtt{dom}(F) \quad r' \notin \mathtt{dom}(\delta_F~F)$
	\begin{align*}
		& \varepsilon, s_1? \vdash \mathtt{load}~ F~ \rhd (Just ~v, (b, Just ~d))\\
		& \varepsilon_{new}, s_1? \vdash \mathtt{load}~ (\delta_F~F)~ \rhd (Nothing,(True, Nothing))
	\end{align*}
	The delta load function will generate $(\leadsto Nothing, \leadsto (True, Nothing))$, which satisfied the Soundness.

	\item
	When $r \notin \mathtt{dom}(F) \quad r' \in \mathtt{dom}(\delta_F~F)$
	\begin{align*}
		& \varepsilon, s_1? \vdash \mathtt{load}~ F~ \rhd (Nothing,(True, Nothing))\\
		& \varepsilon_{new}, s_1? \vdash \mathtt{load}~ (\delta_F~F)~ \rhd (Just ~v, (b, Just ~d))
	\end{align*}
	The delta load function will generate $(\leadsto Just ~v \leadsto (b, Just ~d))$, which satisfied the Soundness.

	\item
	When $r \in \mathtt{dom}(F) \quad r' \in \mathtt{dom}(\delta_F~F)$
	\begin{align*}
		& \varepsilon, s_1? \vdash \mathtt{load}~ F~ \rhd (Just ~v, (b, Just ~d))\\
		& \varepsilon_{new}, s_1? \vdash \mathtt{load}~ (\delta_F~F)~ \rhd (Just ~v', (b', Just ~d'))
	\end{align*}
	By induction, Soundness holds for
	\begin{displaymath}
		\delta_\varepsilon, s_1 \vdash \mathtt{load}_\Delta (F,v,d)~ \delta_F \rhd (\delta_v,\delta_d)
	\end{displaymath}
	\begin{displaymath}
		v' = \delta_v ~ v, \quad d' = \delta_d ~ (b,d)
	\end{displaymath}
	Thus we can conclude that Soundness holds for the optional specification case.
	\end{enumerate}

	\textbf{Case: } $s = [ s \mid x \in e]$\\
	\begin{align*}
		& \varepsilon, s_1? \vdash \mathtt{load}~ F~ \rhd (vs, (b, ds))\\
		& \varepsilon_{new}, s_1? \vdash \mathtt{load}~ (\delta_F~F)~ \rhd (vs', (b', ds'))
	\end{align*}
	For comprehension, we will inspect the change of each single element of $\mathtt{map} ~vs$ and $ds$. We look into these elements by their key value: filename $n$. Then we will prove that the soundness of delete/add/modify each element can eventually composed together to prove the Soundness for updates on the whole map.
	\begin{enumerate}
	\item
	When $ n \in \llbracket e \rrbracket^{\varepsilon_{old}}_{\{Path\}} \setminus \llbracket e \rrbracket^{\varepsilon_{new}}_{\{Path\}}$
	\begin{displaymath}
		(n, vs(n)) \in vs \wedge (n, vs'(n)) \notin vs', (n, ds) \in ds \wedge (n, ds'(n)) \notin ds'
	\end{displaymath}
	Incremental Forest will delete these elements from the $vs'$ and $ds'$, so Soundness holds for this case.


	\item
	When $ n \in \llbracket e \rrbracket^{\varepsilon_{new}}_{\{Path\}} \setminus \llbracket e \rrbracket^{\varepsilon_{old}}_{\{Path\}}$
	\begin{displaymath}
		(n, vs(n)) \notin vs \wedge (n, vs'(n)) \in vs', (n, ds) \notin ds \wedge (n, ds'(n)) \in ds'
	\end{displaymath}
	Incremental Forest will add these elements to the $vs'$ and $ds'$, so Soundness holds for this case.


	\item
	When $ n \in \llbracket e \rrbracket^{\varepsilon_{old}}_{\{Path\}} \cap \llbracket e \rrbracket^{\varepsilon_{new}}_{\{Path\}}$
	\begin{displaymath}
		(n, vs(n)) \in vs \wedge (n, vs'(n)) \in vs', (n, ds) \in ds \wedge (n, ds'(n)) \in ds'
	\end{displaymath}
	Then Incremental Forest will delta load corresponding $v$ and $d$ in the way that satisfies Soundness.
	\begin{displaymath}
		(\delta_\varepsilon, r \mapsto (r, \leadsto r/n)), s \vdash \mathtt{load}_\Delta (F,v,d)~ u \rhd (\delta_v,\delta_d)
	\end{displaymath}
	So that
	\begin{displaymath}
		\delta_v ~vs(n) = vs'(n), \quad \delta_d ~ds(n) = ds'(n)
	\end{displaymath}
	Thus, the modification using $\delta_v$ and $\delta_d$ holds the Soundness for this single element.

	\item
	For these filename categories, we can define them as
	\begin{eqnarray*}
		W_{del} &=& ~\llbracket e \rrbracket^{\varepsilon_{old}}_{\{Path\}} \setminus \llbracket e \rrbracket^{\varepsilon_{new}}_{\{Path\}}\\
		W_{add} &=& ~\llbracket e \rrbracket^{\varepsilon_{new}}_{\{Path\}} \setminus \llbracket e \rrbracket^{\varepsilon_{old}}_{\{Path\}}\\
		W_{keep} &=& ~\llbracket e \rrbracket^{\varepsilon_{old}}_{\{Path\}} \cap \llbracket e \rrbracket^{\varepsilon_{new}}_{\{Path\}}\\
		W_{all} &=& ~\llbracket e \rrbracket^{\varepsilon_{old}}_{\{Path\}} \cup \llbracket e \rrbracket^{\varepsilon_{new}}_{\{Path\}}
	\end{eqnarray*}
	These categories hold the relation that
	\begin{eqnarray*}
		W_{del} \cap W_{add} \cap W_{keep} &=& \emptyset\\
		W_{del} \cap W_{add} \cap W_{keep} &=& W_{all}
	\end{eqnarray*}
	This means that each filename and its data should be updated once and only once correctly. The updates on them won't effect each other, and make every element of the $\mathtt{map}$ deleted or added or become the new data it should be. Thus we can conclude that Soundness holds for the Comprehension specification case.
	\end{enumerate} 

	\textbf{Conclusion}

	By induction, we can conclude that Incremental Forest satisfies Soundness.
\end{proof}

\newpage

\begin{definition}[$\emptyset$ Operation]
We define some $\emptyset$ operations that will produce exactly $\emptyset$ change
\begin{eqnarray}
	\emptyset \cdot \emptyset &=& \emptyset\\
	\fst\emptyset &=& \emptyset\\
	\snd\emptyset &=& \emptyset\\
	\emptyset \otimes \emptyset &=& \emptyset\\
	\mathtt{mod}(n, \emptyset) &=& \emptyset\\
	valid_{12} \leftarrow \emptyset &=& \emptyset\\
	valid_{all} \leftarrow \emptyset &=& \emptyset
\end{eqnarray}
The last two definitions are not graceful as expected. Actually they are unnecessary in real semantic, but we enforce them here to prove Incrementality.
\end{definition}

\begin{theorem}[Incrementality]
If there is no update in file system nor environment, Incremental Forest will return $\emptyset$ update.
\begin{displaymath}
\frac{\begin{array}{c}
	\Delta_{\emptyset} = \{\delta_\varepsilon \mid \delta_\varepsilon \vdash \varepsilon_{old} = \varepsilon_{new}\} \\
	\delta_{\emptyset} \in \Delta_{\emptyset}
\end{array}}
	{\delta_{\emptyset}, r, s \vdash \mathtt{load}_\Delta~ (F,v,d)~ \emptyset \rhd (\emptyset, \emptyset)}
\end{displaymath}

\end{theorem}

\begin{proof}
	By induction\\

	\textbf{Case: } $s = k$
		
		By definition, if $\delta_{\emptyset} \vdash \varepsilon_{old} = \varepsilon_{new}$, Incremental Forest will use the last rule in const specification
		\begin{displaymath}
			\delta_{\emptyset}, r, k \vdash \mathtt{load}_\Delta~ (F,v,d)~ \emptyset \rhd (\emptyset, \emptyset)
		\end{displaymath}

		Incrementality holds for this case.\\

	\textbf{Case: } $s = e::s_1$\\
	
		By definition, if nothing have changed in the environment, we only consider the $\llbracket r/e \rrbracket^{\varepsilon_{old}}_{Path} = \llbracket r/e \rrbracket^{\varepsilon_{new}}_{Path}$ situation
		\begin{displaymath}
		\frac
		{(\delta_{\emptyset} \setminus r, r \mapsto (\llbracket r/e \rrbracket^{\varepsilon_{new}}_{Path}, \emptyset)) , s_1 \vdash \mathtt{load}_\Delta (F,v,d)~ \emptyset \rhd (\delta_v,\delta_d)}
		{\delta_{\emptyset}, e::s_1 \vdash \mathtt{load}_\Delta (F,v,d)~ \emptyset \rhd (\delta_v,\delta_d)}
		\end{displaymath}

		For the new delta environment $\delta_\varepsilon'$,
		\begin{displaymath}
		\frac{
			\delta_\varepsilon' = (\delta_{\emptyset} \setminus r, r \mapsto (\llbracket r/e \rrbracket^{\varepsilon_{new}}_{Path}, \emptyset)) 
			~\wedge~ \delta_{\emptyset} \vdash \varepsilon_{old} = \varepsilon_{new}
		}
		{	\delta_\varepsilon' \vdash \varepsilon_{old}' = \varepsilon_{new}' }
		\end{displaymath}

		This means $\delta_\varepsilon' \in \Delta_{\emptyset}$. By induction,
		\begin{displaymath}
		(\delta_{\emptyset} \setminus r, r \mapsto (\llbracket r/e \rrbracket^{\varepsilon_{new}}_{Path}, \emptyset)) , s_1 \vdash \mathtt{load}_\Delta (F,v,d)~ \emptyset \rhd (\emptyset,\emptyset)
		\end{displaymath}

		Thus, 
		\begin{displaymath}
		\delta_{\emptyset}, e::s_1 \vdash \mathtt{load}_\Delta (F,v,d)~ \emptyset \rhd (\emptyset,\emptyset)
		\end{displaymath}
		
		Incrementality holds for this case.\\

	\textbf{Case: } $s = \langle x : s_1, s_2 \rangle$\\

	By definition,
	\begin{displaymath}
	\frac{\begin{array}{c}
		\delta_{\emptyset}, s_1 \vdash \mathtt{load}_\Delta (F,v_1,d_1)~ \emptyset \rhd (\delta_{v_1},\delta_{d_1})\\
		(\delta_{\emptyset}, x \mapsto (v_1, \delta_{v_1}), x_{md} \mapsto (d_1, \delta_{d_1})), s_2 \vdash \mathtt{load}_\Delta (F,v_2,d_2)~ \emptyset \rhd (\delta_{v_2},\delta_{d_2})\\
		valid_{12}({d_1}',{d_2}') = valid({d_1}') \wedge valid({d_2}')
	\end{array}}
	{\delta_{\emptyset}, \langle x:s_1,s_2 \rangle \vdash \mathtt{load}~ (F,(v_1,v_2),(b,(d_1,d_2)))~ \emptyset \rhd (\delta_{v_1} \otimes \delta_{v_2},valid_{12} \leftarrow (\delta_{d_1} \otimes \delta_{d_2})) }
	\end{displaymath}

	By induction, we have
	\begin{displaymath}
		\delta_{\emptyset}, s_1 \vdash \mathtt{load}_\Delta (F,v_1,d_1)~ \emptyset \rhd (\emptyset,\emptyset)
	\end{displaymath}

	In the second line, we make $\delta_{v_1} = \emptyset $ and $ \delta_{d_1} = \emptyset$.
	\begin{displaymath}
	\frac{
		\delta_\varepsilon' = (\delta_{\emptyset}, x \mapsto (v_1, \emptyset), x_{md} \mapsto (d_1, \emptyset))
	}
	{	\delta_\varepsilon' \vdash \varepsilon_{old}' = \varepsilon_{new}' }
	\end{displaymath}

	Then we have $\delta_\varepsilon' \in \Delta_{\emptyset}$, by induction, 
	\begin{displaymath}
		\delta_\varepsilon', s_2 \vdash \mathtt{load}_\Delta (F,v_1,d_1)~ \emptyset \rhd (\emptyset,\emptyset)
	\end{displaymath}

	By definition, we have
	\begin{displaymath}
		\delta_{\emptyset}, \langle x:s_1,s_2 \rangle \vdash \mathtt{load}~ (F,(v_1,v_2),(b,(d_1,d_2)))~ \emptyset \rhd (\emptyset \otimes \emptyset,valid_{12} \leftarrow (\emptyset \otimes \emptyset)) 
	\end{displaymath}

	Because of the $\emptyset$ operations, finally we have
	\begin{displaymath}
		\delta_{\emptyset}, \langle x:s_1,s_2 \rangle \vdash \mathtt{load}~ (F,(v_1,v_2),(b,(d_1,d_2)))~ \emptyset \rhd (\emptyset, \emptyset) 
	\end{displaymath}

	Incrementality holds for this case.\\

	\textbf{Case: } $s = P(e)$\\

		By definition, when $\varepsilon_{old} = \varepsilon_{new}$,
		\begin{displaymath}
			\frac{ \varepsilon_{old} = \varepsilon_{new}}
			{\delta_\varepsilon, P(e) \vdash \mathtt{load}~ (F,v,d) ~\delta_F \rhd (\emptyset, \emptyset)}
		\end{displaymath}

		Incrementality holds for this case.\\

	\textbf{Case: } $s = s_1?$\\

	By definition, since $\varepsilon_{old} = \varepsilon{new}$, there is only two situations intead of four. When the path doesn't exist in the file system.
	\begin{displaymath}
	\frac{
		r' = \varepsilon_{new}(r) \quad r \notin \mathtt{dom}(F) \quad r' \notin \mathtt{dom}(\delta_F~F)
	}
	{\delta_\varepsilon, s? \vdash \mathtt{load}_\Delta (F,v,d)~ u \rhd (\emptyset, \emptyset)}
	\end{displaymath}

	When the path exists in the file system, by induction
	\begin{displaymath}
	\frac{\begin{array}{c}
	r' = \varepsilon_{new}(r) \quad r \in \mathtt{dom}(F) \quad r' \in \mathtt{dom}(\delta_F~F), \quad
		\delta_{\emptyset}, s \vdash \mathtt{load}_\Delta (F,v,d)~ \delta_F \rhd (\emptyset, \emptyset)
	\end{array}}
	{\delta_\varepsilon, s? \vdash \mathtt{load}_\Delta (F,v,d)~ \delta_F \rhd (\emptyset,\emptyset)}
	\end{displaymath}

	Incrementality holds for this case.\\

	\textbf{Case: } $s = [ s \mid x \in e]$\\

	When $\delta_{\emptyset} \in \Delta_{\emptyset}$, no file is added or deleted in comprehension maps $vs$ and $ds$. Only $mod$ case is considered here. By definition and induction,
	\begin{displaymath}
		\frac{\begin{array}{c}
		\forall n \in \llbracket e \rrbracket^{\varepsilon_{new}}_{\{Path\}} \cap \llbracket e \rrbracket^{\varepsilon_{old}}_{\{Path\}}, ~ v = vs(n), ~d = ds(n)\\
		(\delta_{\emptyset}, r \mapsto (r/n, \emptyset)), s \vdash \mathtt{load}_\Delta (F,v,d)~ u \rhd (\emptyset,\emptyset)
		\end{array}}
		{\mathtt{changeVOf}(n) = \mathtt{mod}((n,v), \snd\emptyset), ~~\mathtt{changeDOf}(n) = \mathtt{mod}((n,d), \snd\emptyset)}
	\end{displaymath}

	By $\emptyset$ operations, changes on every single file is $\emptyset$, thus the record of changes $\Delta_v$ and $\Delta_d$ has the form:
	\begin{eqnarray*}
		\Delta_v &=& \emptyset \cdot \emptyset \dots \cdot \emptyset \quad \Rightarrow \quad \Delta_v = \emptyset\\
		\Delta_d &=& \emptyset \cdot \emptyset \dots \cdot \emptyset \quad \Rightarrow \quad \Delta_d = \emptyset
	\end{eqnarray*}

	Thus,
	\begin{displaymath}
	\frac
	{\delta_{id}, r, [ s \mid x \in e] \vdash \mathtt{load}_\Delta~ (F,vs,ds)~ \emptyset \rhd (\emptyset, valid_{all} \leftarrow \emptyset)}
	{\delta_{id}, r, [ s \mid x \in e] \vdash \mathtt{load}_\Delta~ (F,vs,ds)~ \emptyset \rhd (\emptyset, \emptyset)}
	\end{displaymath}

	Incrementality holds for this case.\\

	\textbf{Conclusion}

	By induction, we can conclude that Incremental Forest satisfies Incrementality.
\end{proof}

\end{document}